Con il termine rischio si vuole identificare la probabilità che un'azione porti al verificarsi di
eventi indesiderati. Nel corso della realizzazione di un progetto i rischi sono inevitabili, è dunque opportuno analizzare i potenziali rischi, stimare la probabilità con cui si possono verificare ed individuare eventuali procedure per mitigarli. \\
\linebreak

\textbf{Identificazione del rischio}\\
L'identificazione del rischio ha l'obiettivo di portare all'attenzione delle parti
coinvolte nel progetto un insieme, il più possibile completo e corretto, di elementi di criticità che
costituiscono la base per la valutazione del rischio generale e per la predisposizione delle
opportune procedure di mitigazione.\\
\linebreak

\textbf{Quantificazione del rischio}\\
La quantificazione dei rischi potenziali dà una base alla valutazione di quella dell'intero
progetto. Si intende evidenziare come la sua valorizzazione non sia assoluta, ma dipenda da due fattori fondamentali: il
tempo e le tecniche adottate per attenuarlo. La prima dipendenza è legata al fatto che al
semplice passare del tempo, gli elementi di rischiosità tendono a cambiare sia in termini
di probabilità di accadimento, che in termini di danno arrecabile al progetto. Per il
secondo aspetto, invece, occorre osservare che l'entità del rischio dipende fortemente
dalle strategie che vengono messe in atto per controllarlo. Possiamo aspettarci
che la valutazione iniziale sia significativamente superiore a quella successiva proprio in virtù dell'efficacia del piano di azione individuato. 

\begin{longtable}{|>{\centering}p{5cm}| >{\centering}m{5cm}| }
    \hline
    \multicolumn{1}{|c|}{\textbf{Livello}} &
    \multicolumn{1}{c|}{\textbf{Probabilità}} \\ %\tabularnewline 
      \hline
        1 & $ \le 5 \% $\tabularnewline
		2 & $ \ge 5 \%$ e $ \le 20 \% $\tabularnewline
		3 & $ \ge 20 \%$ e $ \le 40 \% $\tabularnewline
		4 & $ \ge 40 \%$ e $ \le 60 \% $\tabularnewline
		5 & $ \ge 60 \%$ \tabularnewline
      \hline
    \caption{Metrica valutazione probabilità di rischio}
    \label{tab: Metrica valutazione probabilità di rischio}
  \end{longtable}
  


\textbf{Progettazione interventi}\\
La progettazione ha l'obiettivo di fornire la pianificazione di un insieme coerente e
sostenibile di attività e di responsabilità per il controllo del rischio e l'individuazione
delle modalità di misura associate al controllo dei risultati.\\
\linebreak
\textbf{Esecuzione degli interventi}\\
Avendo stabilito la migliore strategia di mitigazione del rischio, si applica l'intervento
alla lista di processi produttivi interessati.\\
\linebreak
\textbf{Verifica intervento}\\
L'obiettivo di questa attività è di valutare l'efficacia e l'efficienza dimostrata dal piano di
gestione del rischio al fine di confermarne la validità o di innescare una fase di revisione
del sistema di gestione del rischio.


\subsection{Indisposizione del personale} 
\begin{description}
\item[Descrizione:]c'è la possibilità che uno, o più membri del gruppo di lavoro, non possa prendere parte attivamente al progetto per un periodo di tempo variabile causa malattia o impegni personali importanti. Ciò può comportare dei rallentamenti, fino al possibile mancato rispetto dei termini di rilascio previsti.

\item[Soluzione:] In tale circostanza, il gruppo di lavoro sarà considerato in stato d'emergenza. Il responsabile del gruppo o, in mancanza, il PM, ha l'onere di decidere se:
\begin{itemize}
\item l'assente può continuare il proprio lavoro temporaneamente tramite telelavoro;
\item è possibile ridistribuire (tutto o in parte) il lavoro tra i restanti membri del gruppo;
\item si rende necessaria l'assunzione temporanea di nuovo personale;
\item tale assenza comporterà o meno un ritardo del progetto.
\end{itemize}
Se l'indisponibilità tuttavia permane, senza motivi validi, il responsabile dovrà dapprima cercare di risolvere il problema con l'interessato e, in ultima istanza, contattare gli organi sindacali competenti.
\item[Impatto:] 4, se ciò colpisce più membri contemporaneamente, l'impatto può essere gravoso e portare a rallentamenti significativi.
\item[Probabilità:] 3.
\item[Controllo:] Qui sono fondamentali le figure del responsabile e del PM che dovranno monitorare l'avanzamento del progetto ed agire tempestivamente in caso di assenze o ritardi per non causare rallentamenti nello sviluppo del prodotto, è inoltre importante comunicare repentinamente eventuali indisponibilità agli altri membri del gruppo, soprattutto se programmabili, per permettere al responsabile di redistribuire il carico di lavoro correttamente ed in modo efficiente. 
\end{description}

\subsection{Fallimento del progetto}
\begin{description}
\item[Descrizione:] Il prodotto che l'azienda vuole sviluppare è presente sul mercato, ma i numerosi rivali (competitor) che propongono questo tipo di prodotto si sono cimentati solamente nello sviluppo dell'applicativo per piattaforma Microsoft senza dare importanza a altre piattaforme come Linux.
Il rischio è quindi che i competitor aggiornino i loro prodotti includendo anche le
funzionalità da noi proposte(sviluppo per più piattaforme) e, forti di una base di utenti molto ampia, non permettano ulteriori sviluppi del progetto decretandone una fine prematura. La sola presenza nel mercato di altri competitor con un'ampia base di utenza, pone il progetto a rischio di fallimento.
\item[Soluzione:] Per cercare di minimizzare il più possibile tale rischio si adotteranno le
seguenti procedure:
\begin{itemize}
\item nelle fasi precedenti all'analisi e alla progettazione del prodotto sarà posta
massima cura nel pregi e difetti delle soluzioni adottate dai competitor, oltre ad evidenziarne eventuali parti
migliorabili. Sarà posta anche particolare cura nel capire quali sono le mind maps degli utenti che utilizzano prodotti affini a quello sviluppato e si cercherà di replicarle nel prodotto finito. Tale strategia permette all'azienda di partire avvantaggiata nello sviluppo del prodotto, cercando di replicare il feeling che l'utente ha sviluppato con prodotti affini e cercando di migliorare i difetti comuni.
\item nelle fasi immediatamente precedenti al rilascio sarà fatta una campagna
pubblicitaria massiva e che punti a far crescere l'interesse verso il nostro prodotto.
\item nelle fasi successive al rilascio la campagna pubblicitaria dovrà rendere chiaro
in cosa il prodotto si distingue rispetto ai concorrenti.
\item l'azienda dovrà porre particolare cura al rapporto con gli utenti e dovrà
proporre soluzioni o migliorie tempestive basandosi sui possibili consigli da parte del pubblico.
\item l'azienda in generale dovrà mettere in atto quanto descritto nella sezione Pubblicità  del documento
\end{itemize}
\item[Impatto:] 5, la mancata adozione del nostro prodotto da parte degli utenti porterebbe
al fallimento stesso del progetto.
\item[Probabilità:] 3, il numero di competitor non è molto elevato ed il rischio che loro adattino
il loro prodotto includendo le novità da noi proposte è sostanziale. 
\item[Controllo:] Il PM deve monitorare attentamente gli indicatori di utilizzo del prodotto
e valutare settimanalmente quali azioni compiere per incrementarli.
\end{description}

\subsection{Difficoltà nel pianificare correttamente i tempi di sviluppo}
\begin{description}
\item[Descrizione:] C'è il forte rischio di effettuare una stima errata in materia di ore/persona necessarie allo svolgimento dell'intero progetto e della loro allocazione alle rispettive figure professionali in gioco. Le ore schedulate infatti potrebbero essere
insufficienti o allocate in ruoli in cui ne sono sufficienti un numero minore. Il mancato utilizzo
di un modello di sviluppo lungimirante, potrebbe portare a una serie di
iterazioni dannose per il progetto. Il rischio, quindi, e quello di non rispettare una
delle milestone previste per il progetto e di perdere ulteriore strada rispetto alla concorrenza.
\item[Soluzione:] Nella pianificazione delle milestone si è tenuto conto di questo rischio e si è
cercato di pensare a tutti i possibili fattori che potrebbero porre problemi al riguardo.
\item[Impatto:] 4, forti errori potrebbero portare seri problemi al gruppo e alla realizzazione del progetto.
\item[Probabilià:] 5, problematica praticamente inevitabile.
\item[Controllo:] Fondamentale è il ruolo del PM e degli altri responsabili, che devono sempre
tenere monitorato l'avanzamento del proprio lavoro e reindirizzare le risorse
dove maggiormente necessarie al fine di rispettare le milestone previste, con gli
obiettivi di qualità posti.
\end{description}

\subsection{Analisi dei requisiti non corretta}
\begin{description}
\item[Descrizione:] Fondamentale per il successo del progetto è un'analisi dei requisiti di
qualità ed uno studio molto approfondito del bisogno degli utenti target e delle
caratteristiche positive e negative dei competitor analizzati. Un errore in questa fase
del progetto sarebbe identificato solamente nei primi test pubblici del prodotto,
il che significherebbe dover riprendere per mano l'intero processo di sviluppo con
significative perdite di tempo e risorse.
\item[Soluzione:] La ripartizione oraria terrà conto della necessità di assegnare un numero
sostanzioso di ore al processo di analisi e sarà utilizzato, ove necessario,
personale esterno altamente qualificato al fine di garantire una buona riuscita di
questo processo così critico. 
\item[Impatto:] 5, un processo di analisi completato frettolosamente potrebbe portare a gravi rallentamenti ed anche al fallimento del progetto.
\item[Probabilità:] 2, nonostante le misure intraprese per evitare questo problema, saranno sicuramente possibili delle problematiche relative ai requisiti.
\item[Controllo:] Il PM  accertarsi che quanto prodotto durante il processo di analisi sia in linea con la visione aziendale; tuttavia sarà solo nelle fasi di test da parte degli utenti che si potranno delineare eventuali problematiche.
\end{description}

\subsection{Cause legali}
\begin{description}
\item[Descrizione:] Vista la tipologia di prodotto sviluppato e vista la grande quantità di
dati personali memorizzati, è possibile che un competitor o un gruppo di utenti, decidano di fare causa all'azienda.
\item[Soluzione:] Per evitare che i competitor agiscano legalmente contro l'azienda bisogna porre massima cura nell'accertarsi che le idee o i software utilizzati non vengano usati in maniera illegale e che i brevetti di terzi siano rispettati. Per
quanto riguarda le azioni legali intraprese dagli utenti si possono evidenziare 2
motivazioni:
\begin{itemize}
\item privacy: in questo caso basterà informare preventivamente gli utenti di come
i loro dati verranno salvati all'interno del database.
\item  altro: un utente potrebbe far causa all'azienda per le motivazioni più svariate
(es.: tematiche affrontate in certi progetti, commenti fatti dai dipendenti
dell'azienda, ecc); in questo caso sarà il RL a valutare il miglior modo di
gestire la situazione.
\end{itemize}
\item[Impatto:] 5, una causa legale può protrarsi per per molto tempo e portare ad uno spreco di finanze
dell'azienda mettendo a rischio il progetto stesso.
\item[Probabilità:] 2, bisognerà stare molto attenti essendo una problematica spesso presente.
\item[Controllo:] Il RL dovrà definire attentamente i contratti e tutta la documentazione
legale. Il PM dovrà chiamarlo prontamente in causa alle prime avvisaglie di
problematiche legali, in modo da poterle risolvere nel minor tempo possibile.
\end{description}

