Il progetto consta nella realizzazione di un software per computer desktop in grado di essere utilizzabile su diversi sistemi operativi tra i quali \emph{Microsoft Windows}, \emph{Apple Mac OSX} e \emph{GNU}\textbackslash\emph{Linux}.\\
Per ottenere tale risultato si è scelto di utilizzare \emph{Java}\footnote{\url{http://it.wikipedia.org/wiki/Java\_\%28linguaggio\_di\_programmazione\%29}}, un linguaggio di programmazione multipiattaforma semi-interpretato che consente di scrivere programmi che vengono eseguiti su di una macchina virtuale installabile sui tutti i sistemi operativi sopracitati.\\
Il progetto, per rispondere ai requisiti, deve risolvere 3 problematiche principali che sono a loro volta legate a 3 differenti tecnologie informatiche:
\begin{itemize}
\item \textbf{Accesso ai server di posta:} per recuperare le email contenti le newsletter di interesse, si deve poter accedere ai server di posta che le contengono; esistono innumerevoli provider che forniscono tale servizio agli utenti, sia per uso personale e gratuiti, che per uso professionale e a pagamento.\\
Per non porre limiti nell'utilizzo del software di dovrà quindi sviluppare un sistema che sia in grado di connettersi indipendentemente a qualsiasi server di posta: per ottenere questo risultato si dovrà utilizzare la \emph{JavaMail API}\footnote{\url{http://docs.oracle.com/javaee/7/api/}} presente  nella \emph{Java Enterprise Edition}.
\item \textbf{Lettura della newsletter:} le email strutturate, come possono essere le newsletter, sono composte mediante il linguaggio di markup \emph{HTML}; questo consente di leggere molto facilmente il contenuto dei messaggi ma pone un problema da non sottovalutare: infatti tale linguaggio, pur essendo standard, non fornisce un modo universale di strutturare le informazioni, ma lascia al programmatore ampia libertà di scrittura.\\
Con alta probabilità ogni newsletter proveniente da un'azienda avrà una struttura molto diversa da quella proveniente da una seconda azienda; per questo motivo si è deciso di personalizzare il prodotto in base alle esigenze di ogni singolo cliente.\\
Lo strumento che consentirà di analizzare in modo efficiente le diverse tipologie di newsletter è la libreria \emph{Jsoup}\footnote{\url{http://jsoup.org/}}. Libreria open source scritta in Java e quindi facilmente utilizzabile all'interno del progetto.
\item \textbf{Salvataggio delle informazioni nel database:} dopo aver ottenuto le informazioni da una newsletter si presenta la necessità di conservarle in modo persistente nella memoria di massa del personal computer del cliente.\\
Il modo migliore di salvare tali informazioni è quello di utilizzare un database dal quale si potranno, in un secondo momento, recuperare tutte le notizie volute impostando ricerche specifiche a discrezione dell'utente.\\
Per realizzare la base di dati si è scelto \emph{Oracle MySQL}\footnote{\url{http://it.wikipedia.org/wiki/MySQL}}, DBMS open source disponibile per i 3 sistemi operativi di riferimento, in quanto sono disponibili driver JDBC che consentono un'agevole gestione delle connessioni al database e delle relative query.
\end{itemize}